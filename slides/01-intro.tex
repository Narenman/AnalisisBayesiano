\documentclass[10pt]{beamer}

\usetheme{u18fest}
\usepackage{url}

\subtitle{Marco Bayesiano para el análisis de datos,\\
calibración de parámetros y modelamiento inverso}
\title{Introducción}
\institute{Universidad Industrial de Santander}
\date{U18 Fest}

\begin{document}

\begin{frame}[noframenumbering]
  \titlepage
\end{frame}

\begin{frame}
  \frametitle{Introducción}

  \textbf{David A. Barajas-Solano}
  \begin{itemize}
  \item Egresado de la UIS -- Ingeniería Civil
  \item Ph.D. en ciencias de la ingeniería\\(University of California, San Diego)
  \item Científico en Pacific Northwest National Laboratory
  \item Perfil científico en \url{https://www.dbarajassolano.com}
  \end{itemize}
  
\end{frame}
%
\begin{frame}
  \frametitle{Programa}
  \begin{tcolorbox}[title=Fundamentos]
    \begin{itemize}
    \item \emph{Lunes}: Teoría de la probabilidad y teorema de Bayes
    \item \emph{Martes}: Modelamiento y programación probabilística
    \end{itemize}
  \end{tcolorbox}
  \begin{tcolorbox}[title=Regresión]
    \begin{itemize}
    \item \emph{Martes}: Modelos lineales
    \item \emph{Miércoles}: Modelos no lineales
    \item \emph{Jueves}: Modelos jierárquicos
    \end{itemize}
  \end{tcolorbox}
\end{frame}
%
\begin{frame}
  \frametitle{Filosofía de cálculo}
  \begin{itemize}
  \item Desenfoque a software de oficina
  \item Enfoque a interfaces de línea de comandos (CLI) y texto plano
  \end{itemize}
  
  \textbf{Porqué?}
  \begin{itemize}
  \item Reproducibilidad
  \item Transparencia
  \item Mantenibilidad
  \end{itemize}
  \pause
  \begin{tcolorbox}[title=En éste taller]
    \begin{itemize}
    \item Cuadernos de cálculos de Jupyter
    \item Python y PyMC3
    \end{itemize}
  \end{tcolorbox}
  \textbf{Preguntas bienvenidas a cualquier momento!}
\end{frame}
%
\begin{frame}
  \frametitle{Logística}
  \begin{itemize}
  \item Cronograma
    \begin{itemize}
    \item Dos mitades de 1h 20m
    \item Pausa de 20m entre mitades
    \end{itemize}
  \item Código, cuadernos y diapositivas disponibles en GitHub\\
    \url{https://github.com/dbarajassolano/u18fest}
  \item \textbf{Preguntas bienvenidas a cualquier momento!}
  \end{itemize}
\end{frame}

\end{document}

%%% Local Variables:
%%% mode: latex
%%% TeX-master: t
%%% TeX-engine: xetex
%%% End:
